\documentclass[a4paper,10pt]{article}
\usepackage[utf8]{inputenc}
\usepackage{listings}

\title{Rechnernetze Aufgabe 1 Konzept}

\author{
  Triebe, Marian\\
  \texttt{marian.triebe@haw-hamburg.de}
  \and
  Kirstein, Katja\\
  \texttt{katja.kirstein@haw-hamburg.de}
}

\begin{document}

\maketitle

\section{Aufgabenbeschreibung}

Es soll ein Server-Programm entwickelt werden, welches wie ein Multiplexer Bilder weiterleitet.
Diese Bilder stammen von einer Kamera. Die einzelnen Bilder bestehen aus 320 x 240 Pixeln.
Für jeden Pixel werden 3 Byte übertragen (RGB). Es soll möglich sein mindestens 20 Bilder die Sekunde
zu übertragen, desweiteren soll das Programm die empfangenen Bilder
der Kamera ausgeben. Die Kommunikation soll mit dem TCP/IP Protokoll abgehandelt werden. Mögliche
Angriffsvektoren sind abzusichern. Der Server soll nach fehlerhaften Verhalten der Clienten möglichst
ohne Störung weiterarbeiten.

\section{Aufbau des Programms}

Die Anwendung soll aus 4 Hauptklassen bestehen. Jeder dieser Hauptklassen implementiert unseren pthread Wrapper
(thread\_t.hpp). Insgesamt wird es also in der Anwendung mindestens 4 Threads geben (inklusive Hauptthread).
Jede aktive Verbindung zu einem Client wird jedoch durch einen zusätzlichen Thread abgearbeitet.
Folgende Aufgaben sind auf diese 4 Hauptklassen aufzuteilen:
\begin{enumerate}
 \item Abholen der Bild Daten (fetch\_impl.hpp)
 \item Warten auf Verbindungsanfragen von Clients (server\_impl.hpp)
 \item Senden der Bild Daten zu Clients (worker\_impl.hpp)
 \item Ausgabe der Bilder am lokalen Rechner (client\_impl.hpp)
\end{enumerate}

\section{Buffer Konzept}

Die Bufferung soll mittels eines Ringbuffers gehandhabt werden. Für die Größe des Ringbuffers wollen wir erst einmal
100 Bilder verwenden. Der Ringbuffer soll eine Funktion `wait\_on\_picture()` anbieten, welche den Aufrufer blockieren lässt,
bis ein neues Bild im Buffer zur verfügung steht. Der Ringbuffer hat immer eine konstante Größe und kann somit nicht von
langsam sendenden workern (bzw. langsam empfangenden Clients)  aufgeblasen werden. Nachteil hierbei, es können Bilder verloren gehen.

\section{Angriffsvektoren}

\begin{itemize}
 \item Überflutung des Servers mit connect() anfragen
 \item Clients lesen nur sehr kleine Datensätze
 \item Clients warten lange zwischen dem empfangen der einzelnen Bilder
\end{itemize}

\end{document}
