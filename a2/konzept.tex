\documentclass[a4paper,10pt]{article}
\usepackage[utf8]{inputenc}
\usepackage{listings}

\title{Rechnernetze Aufgabe 2 Konzept}

\author{
  Triebe, Marian\\
  \texttt{marian.triebe@haw-hamburg.de}
  \and
  Kirstein, Katja\\
  \texttt{katja.kirstein@haw-hamburg.de}
}

\begin{document}

\maketitle

\section{Aufgabenbeschreibung}
Mit Hilfe von Routingtabellen sollen verschiedene Netzwerkeigenschaften getestet werden,
dazu geh\"oren die Verz\"ogerungszeiten sowie die maximale Transferrate.
Zum analysieren der Verz\"ogerungszeit kann der Befehl `ping` verwendet werden.
Die Paketgr\"o{\ss}en sollen zwsichen 50 und 3000 variieren. Aus den gemessenen Zeiten
soll ein Diagramm erstellt werden.

\section{\"Ubertragung von großen Datenmengen}
Gr\"o{\ss}ere Datenmengen k\"onnen mit Hilfe der Programme `netperf` und `netserver`
\"ubertragen werden, die Programme arbeiten mit dem TCP Protokoll. Mit Hilfe von
`wireshark` soll der Verkehr mitgeschnitten werden. Die mitgeschnittenen Daten k\"onnen
dann mit Hilfe des `TCPAnalyser` visualisiert werden. Anhand der Visualisierung sollen
Datenrate sowie RTT ermittelt werden. Zu kl\"aren ist die Frage warum die RTT nicht mit der
zuvor erstellten Messung \"ubereinstimmt.

\section{Messroute}
Alle Routen sollen \"uber das Labornetzwerk (192.138.17.0 bzw. 192.168.18.0) f\"uhren.
Es sind drei Messrouten zu konfigurieren und zu messen.
\begin{itemize}
 \item Direkte Verbindung zweier Rechner \"uber einen Switch
 \item Verbindung \"uber einen Router
 \item WAN-Verbindung (Wide Area Network)
\end{itemize}

\newpage

\section{Konfiguration einer Messroute}
Die Routen im Netzwerk k\"onnen mit dem Befehl `route` gesetzt werden.
\begin{itemize}
 \item Anlegen einer neuen Route: `route add -net 192.160.0.0/16 gw 192.159.1.1`,
 in diesem Beispiel ist das Zielnetzwerk 192.160.0.0/16 und der Gateway ist 192.159.1.1.
 \item L\"oschen einer Route: `route del -net Ziel/CIDR gw Gateway-Adresse`
 \item Alle konfigurierten Routen anzeigen: `route` oder `route -n`
\end{itemize}

\section{Ablauf einer Messung}
Zu den verschiedenen Messrouten gibt es folgende Fragen zu kl\"aren
\begin{enumerate}
 \item Direkte Verbindung zweier Rechner \"uber einen Switch
 \begin{itemize}
  \item Wie hoch ist die RTTI in Abhängigkeit von der Paketgröße?
  Stimmt das mit den Erwartungen überein?
  \item Wie hoch ist die maximale Transferrate bei der Übertragung größerer Datenmengen
  mit TCP? Entspricht das den Erwartungen? Kann man Auswirkungen der Algorithmen zur Stauvermeidung erkennen?
  \item Kann man Aussagen treffen zur Arbeitsweise der Switch?
  \item Hat die TCP-Fenstergröße einen Einfluss?
 \end{itemize}
 \item Verbindung über einen Router
 \begin{itemize}
  \item Welchen Einfluss hat die Routerperformance auf die RTTI?
  \item Wie verändert sich der Durchsatz?
  \item Welche Eigenschaften haben die Schnittstellen zum Router?
 \end{itemize}
 \item WAN-Verbindung (Wide Area Network)
 \begin{itemize}
  \item Bestimmen Sie die RTTI
  \item Berechnen Sie die Länge der simulierten Strecke
  \item Bestimmen Sie die maximale Transferrate
  \item Ermitteln Sie die Mindestgröße des TCP-Fensters, damit eine vollständige Ausnutzung der Leitung erfolgt.
  Führen Sie hierfür Messungen mit unterschiedlichen Fenstergrößen durch. Tragen Sie die erzielte Transferrate als Funktion von der Fenstergröße auf.
  Überprüfen Sie die Richtigkeit Ihrer Messungen
  \item Kennzeichnen Sie in den Diagrammen die verschiedenen Phasen der Stauvermeidung
 \end{itemize}
\end{enumerate}

\section{Programme und Befehle}
\begin{itemize}
 \item netserver
 \begin{itemize}
  \item L\"auft als Daemon im Hintergrund und wartet auf eingehende Verbindungen (TCP Port 12865)
  \item F\"ur das starten sind keine Parameter notwendig
  \item Dient als Gegenstelle auf dem entfernten Rechner
 \end{itemize}
 \item netperf
 \begin{itemize}
  \item Sendet Daten an den `netserver` Daemon
  \item `netperf -H ziel` spezifiziert den Zielrechner
  \item Beispieleingabe `netperf -H 192.168.1.1 -l -200000 -- -S 65536`
 \end{itemize}
\end{itemize}

\section{Absch\"atzung der erwarteten Ergebnisse}
Wir erwarten sehr niedrige RTT Zeiten, da Strecke die in dem lokalen Netzwerk zur\"uckgelegt
werden muss eher gering ist. Sollten sich die Messwerte von `ping` sowie `netserver/netperf` zu
sehr unterscheiden, wird das wahrscheinlich am von `ping` verwendeten ICMP Protokoll liegen.
Desweiteren sch\"atzen wir, dass der WAN Aufbau sowie die Verbindung \"uber den Router die schlechteste
Netzwerkleistung zeigen werden.
\newline

\textit{Fragestellungen der einzelnen Teilaufgaben,
sowie Beschreibung der Programme sind der Aufgabenstellung entnommen}

\end{document}
